These networks are state of the art examples of what can be implemented on different datasets. 

In table \ref{table:summary_SoA_nets}, we present an overview of each standard network, then we present all described nets in more details. 


\begin{table}[h]
\begin{center}
 \begin{tabular}{|l|l||c||c|c|c|}
  \hline
  &                                          &                           & \multicolumn{3}{c|}{Number of weights}        \\
  \cline{4-6}
  & Net                                      & Input image size          & Convolutions & Fully-connected & Total       \\
  \hline
  \multirow{3}{*}{Small}
  & LeNet \cite{MNIST}                       & $28 \times 28 \times 1$   & 25,500       & 405,000         & 430,500     \\ 
  & CIFAR-10 \cite{CIFAR}                    & $32 \times 32 \times 3$   & 79,200       & 66,176          & 145,376     \\
  & LR-CNN \cite{Chevalier15}                & $32 \times 32 \times 3$   &  &  & \\
  \hline
  \multirow{3}{*}{Medium}
  & AlexNet \cite{Krizhevsky12}              & $227 \times 227 \times 3$ & 2,332,704    & 58,621,952      & 60,954,656 \\
  & Overfeat large \cite{Sermanet14overfeat} & $221 \times 221 \times 3$ &  &  & \\
  & CNN-M \cite{Chatfield14}                 & $224 \times 224 \times 3$ & 6,526,752 & 96,370,688 & 102,897,440 \\
  \hline
  \multirow{2}{*}{Large}
  & Very deep                                &                           &  &  & \\ 
  & GoogLeNet                                &                           &  &  & \\
  \hline
 \end{tabular}
 \caption{State of the art CNN architectures - summary.}
 \label{table:summary_SoA_nets}
\end{center}
\end{table}









\section{Small networks}

\subsection{LeNet (on MNIST)}

LeNet architecture \cite{MNIST}, described in table \ref{table:LeNet} was initially designed for MNIST image classification, and contains a total of 
\textbf{430,500} learnable weights (25,500 in the convolutional part, 405,000 in the fully-connected part). 

MNIST contains 70,000 images of handwritten digits. These images belong to 10 classes (``0'', ``1'' ... ``9'') and have a size $28 \times 
28 \times 1$ (greyscale images). 

\begin{table}[h]
\begin{center}
 \begin{tabular}{|l||l||l|c|c||l|}
   \hline
   Layer name  & Input                          & Details                                 & Stride & Padding & Output                 \\
   \hline
   \hline
   conv1       & $28 \times 28 \times 1$ images & 20 filters $5 \times 5 \times 1$        & 1      & 0       & 20 maps $24 \times 24$ \\
   pool1       & 20 maps $24 \times 24$         & max on $[2, 2]$ regions                 & 2      & 0       & 20 maps $12 \times 12$ \\
   \hline
   conv2       & 20 maps $12 \times 12$         & 50 filters $5 \times 5 \times 20$       & 1      & 0       & 50 maps $8 \times 8$   \\
   pool2       & 50 maps $8 \times 8$           & max on $[2, 2]$ regions                 & 2      & 0       & 50 maps $4 \times 4$   \\
   \hline
   fc3         & 50 maps $4 \times 4$           & 500 neurons                             &        &         & vector of size 500     \\
   relu3       & vector of size 500             &                                         &        &         & vector of size 500     \\
   fc4         & vector of size 500             & 10 neurons                              &        &         & vector of size 10      \\
   \hline
   sotfmax     & vector of size 10              &                                         &        &         & vector of size 10      \\
   log-loss    & vector of size 10              & only for training                       &        &         & loss value             \\
   \hline
 \end{tabular}
 \caption{Architecture example on MNIST (architecture: LeNet \cite{MNIST}). Total weights: 430,500.}
 \label{table:LeNet}
\end{center}
\end{table} 




\subsection{Standard architecture on CIFAR}

The architecture described in table \ref{table:CIFAR} is a standard architecture used on CIFAR-10 image classification, and contains a total of 
\textbf{145,376} learnable weights (79,200 in the convolutional part, 66,176 in the fully-connected part). 

CIFAR \cite{CIFAR} contains  images centered on objects. These images belong to 10 classes (CIFAR-10) or 100 classes (CIFAR-100) and have a size $32 
\times 32 \times 3$. 

\begin{table}[h]
\begin{center}
 \begin{tabular}{|l||l||l|c|c||l|}
   \hline
   Layer name & Input                          & Details                                 & Stride & Padding   & Output                 \\
   \hline
   \hline
   conv1      & $32 \times 32 \times 3$ images & 32 filters $5 \times 5 \times 3$        & 1      & [2,2;2,2] & 32 maps $32 \times 32$ \\
   pool1      & 32 maps $32 \times 32$         & max on $[3, 3]$ regions                 & 2      & [0,1;0,1] & 32 maps $16 \times 16$ \\
   relu1      & 32 maps $16 \times 16$         &                                         &        &           & 32 maps $16 \times 16$ \\
   \hline
   conv2      & 32 maps $16 \times 16$         & 32 filters $5 \times 5 \times 32$       & 1      & [2,2;2,2] & 32 maps $16 \times 16$ \\
   relu2      & 32 maps $16 \times 16$         &                                         &        &           & 32 maps $16 \times 16$ \\
   pool2      & 32 maps $16 \times 16$         & average on $[3, 3]$ regions             & 2      & [0,1;0,1] & 32 maps $8 \times 8$   \\
   \hline
   conv3      & 32 maps $8 \times 8$           & 64 filters $5 \times 5 \times 32$       & 1      & [2,2;2,2] & 64 maps $8 \times 8$   \\
   relu3      & 64 maps $8 \times 8$           &                                         &        &           & 64 maps $8 \times 8$   \\
   pool3      & 64 maps $8 \times 8$           & average on $[3, 3]$ regions             & 2      & [0,1;0,1] & 64 maps $4 \times 4$   \\
   \hline
   fc4        & 64 maps $4 \times 4$           & 64 neurons                              &        &           & vector of size 64      \\
   fc5        & vector of size 64              & 10 neurons                              &        &           & vector of size 10      \\
   \hline
   sotfmax     & vector of size 10              &                                         &        &         & vector of size 10      \\
   log-loss    & vector of size 10              & only for training                       &        &         & loss value             \\
   \hline
 \end{tabular}
 \caption{Architecture example on CIFAR-10 \cite{CIFAR}. Total weights: 145,376.}
 \label{table:CIFAR}
\end{center}
\end{table} 







\subsection{LR-CNN}

This architecture was proposed in \cite{Chevalier15}. Adapted from the historic AlexNet-like networks, this architecture is designed to recognize 
low resolution images in a fine-grained context. More precisely, this structure has been tested on FGVC-Aircraft images \cite{maji13fine-grained} 
resized to a $32 \times 32 \times 3$ input size. 
This dataset contains 10,000 images belonging to 100 aircraft classes. 

The proposed architecture contains a total of  weights ( in the convolutional part, in the fully-connected part).


\begin{table}[h]
\begin{center}
 \begin{tabular}{|l||l||l|c|c||l|}
   \hline
   Layer name & Input                          & Details                                 & Stride & Padding   & Output                 \\
   \hline
   \hline
   conv1      & $32 \times 32 \times 3$ images & 64 filters $5 \times 5 \times 3$        & 1      & [2,2;2,2] &  \\
   relu1      &                                &                                         &        &           &  \\
   norm1      &                                &                                         &        &           &  \\
   pool1      &                                & max on $[3,3]$ regions                  & 2      & [1,1;1,1] &  \\
   \hline
   conv2      &                                & 64 filters $5 \times 5 \times 64$       & 1      & [2,2;2,2] &  \\
   relu2      &                                &                                         &        &           &  \\
   pool2      &                                & max on $[3,3]$ regions                  & 2      & [1,1;1,1] &  \\
   \hline
   conv3      &                                & 64 filters $3 \times 3 \times 64$       & 1      & 0         &  \\
   relu3      &                                &                                         &        &           &  \\
   pool3      &                                & max on $[3,3]$ regions                  & 2      & [1,1;1,1] &  \\
   \hline
   fc4        & 64 maps of size $3 \times 3$   & 128 neurons                             &        &           & vector of size 128 \\
   relu4      & vector of size 128             &                                         &        &           & vector of size 128 \\
   \hline
   fc5        & vector of size 128             & 100 neurons                             &        &           & vector of size 100 \\
   softmax    & vector of size 100             &                                         &        &           & vector of size 100 \\
   log-loss   & vector of size 100             & only for training                       &        &           & loss value         \\
 \end{tabular}
 \caption{LR-CNN architecture \cite{Chevalier15}. Total weights: .}
 \label{table:LR-CNN}
\end{center}
\end{table} 







%===========================================================================================================================================


\section{Medium networks}



\subsection{AlexNet (Krizhevsky)}
\label{AlexNet}

With their paper \cite{Krizhevsky12}, Krizhevsky et al. have achieved tremendous performances on the ILSVRC12 image classification challenge. This 
paper has definitely caused a re-orientation of all image recognition works towards Convolutional Neural Networks. 

This networks contains a total of \textbf{60,954,656} learnable weights (2,332,704 in the convolutional part, 58,621,952 in the fully-connected part).

In table \ref{table:AlexNet}, we describe the architecture in detail. 
It should be noted that, because of the limited computational material at the time of the publication, the network had to be learned in two threads. 
More precisely, we can see in table \ref{table:AlexNet} that for layers conv4 and conv5, the depth of the filters is \textit{half} the depth of the 
input maps. Indeed, half of those filters were learned on a first GPU, and the other half were learned on a second GPU. Both threads converge then so 
that the fully-connected layers are learned jointly. For more details, see \cite{Krizhevsky12}. 




\begin{table}[h]
\begin{center}
 \begin{tabular}{|l||l||l|c|c||l|}
   \hline
   Layer name & Input                            & Details                                 & Stride & Padding   & Output                   \\
   \hline
   \hline
   conv1      & $227 \times 227 \times 3$ images & 96 filters $11 \times 11 \times 3$      & 4      & 0         & 96 maps $55 \times 55$   \\
   relu1      & 96 maps $55 \times 55$           &                                         &        &           & 96 maps $55 \times 55$   \\
   pool1      & 96 maps $55 \times 55$           & max on $[3,3]$ regions                  & 2      & 0         & 96 maps $27 \times 27$   \\
   norm1      & 96 maps $27 \times 27$           & contrast normalization                  &        &           & 96 maps $27 \times 27$   \\
   \hline
   conv2      & 96 maps $27 \times 27$           & 256 filters $5 \times 5 \times 48$      & 1      & [2,2;2,2] & 256 maps $27 \times 27$  \\
   relu2      & 256 maps $27 \times 27$          &                                         &        &           & 256 maps $27 \times 27$  \\
   pool2      & 256 maps $27 \times 27$          & max on $[3,3]$ regions                  & 2      & 0         & 256 maps $13 \times 13$  \\
   norm2      & 256 maps $13 \times 13$          & contrast normalization                  &        &           & 256 maps $13 \times 13$  \\
   \hline
   conv3      & 256 maps $13 \times 13$          & 384 filters $3 \times 3 \times 256$     & 1      & [1,1;1,1] & 384 maps $13 \times 13$  \\
   relu3      & 384 maps $13 \times 13$          &                                         &        &           & 384 maps $13 \times 13$  \\
   \hline
   conv4      & 384 maps $13 \times 13$          & 384 filters $3 \times 3 \times 192$     & 1      & [1,1;1,1] & 384 maps $13 \times 13$  \\
   relu4      & 384 maps $13 \times 13$          &                                         &        &           & 384 maps $13 \times 13$  \\
   \hline
   conv5      & 384 maps $13 \times 13$          & 256 filters $3 \times 3 \times 192$     & 1      & [1,1;1,1] & 256 maps $13 \times 13$  \\
   relu5      & 256 maps $13 \times 13$          &                                         &        &           & 256 maps $13 \times 13$  \\
   pool5      & 256 maps $13 \times 13$          & max on $[3,3]$ regions                  & 2      & 0         & 256 maps $6 \times 6$    \\
   \hline
   fc6        & 512 maps $6 \times 6$            & 4,096 neurons                           &        &           & vector of size 4,096     \\
   relu6      & vector of size 4,096             &                                         &        &           & vector of size 4,096     \\
   \hline
   fc7        & vector of size 4,096             & 4,096 neurons                           &        &           & vector of size 4,096     \\
   relu7      & vector of size 4,096             &                                         &        &           & vector of size 4,096     \\
   \hline
   fc8        & vector of size 4,096             & 1,000 neurons                           &        &           & vector of size 1,000     \\
   softmax    & vector of size 1,000             &                                         &        &           & vector of size 1,000     \\
   log-loss   & vector of size 1,000             & only for training                       &        &           & loss value               \\
   \hline 
 \end{tabular}
 \caption{AlexNet architecture \cite{Krizhevsky12}. Total weights: 60,954,656.}
 \label{table:AlexNet}
\end{center}
\end{table} 







\subsection{Overfeat}

Overfeat is a deep architecture proposed in \cite{Sermanet14overfeat} to classify images from various standard datasets (PASCALVOC, ImageNet...). The 
structure is described in table \ref{table:Overfeat}. The net contains a total of \textbf{} learnable weights ( in the convolutional part,  in 
the fully-connected part).

Overfeat can easily be used as a feature extractor, since the pretrained weights are available online\footnote{Overfeat webpage: 
\url{http://cilvr.nyu.edu/doku.php?id=software:overfeat:start}}. 
Two variants of architectures have been proposed: a fast version (quite fast to compute but not the most accurate) and a large version (the most 
accurate but the slowest version). Here, we detail the large version. 

\begin{table}[h]
\begin{center}
 \begin{tabular}{|l||l||l|c|c||l|}
   \hline
   Layer name & Input                            & Details                                 & Stride & Padding   & Output                   \\
   \hline
   \hline
   conv1      & $221 \times 221 \times 3$ images & 96 filters $7 \times 7$                 & 2      &           & 96 maps                  \\
   relu1      &                                  &                                         &        &           &                          \\
   pool1      &                                  & max on $[3,3]$ regions                  & 3      & 0         & 96 maps $36 \times 36$   \\
   \hline
   conv2      & 96 maps $36 \times 36$           & 256 filters $7 \times 7$                & 1      &           & 256 maps                 \\
   relu2      &                                  &                                         &        &           &                          \\
   pool2      &                                  & max on $[2,2]$ regions                  & 2      & 0         & 256 maps $15 \times 15$  \\
   \hline
   conv3      & 256 maps $15 \times 15$          & 512 filters $3 \times 3$                & 1      & [1,1;1,1] &                          \\
   relu3      &                                  &                                         &        &           & 512 maps $15 \times 15$  \\
   \hline
   conv4      & 512 maps $15 \times 15$          & 512 filters $3 \times 3$                & 1      & [1,1;1,1] &                          \\
   relu4      &                                  &                                         &        &           & 512 maps $15 \times 15$  \\
   \hline
   conv5      & 512 maps $15 \times 15$          & 1,024 filters $3 \times 3$              & 1      & [1,1;1,1] &                          \\
   relu5      &                                  &                                         &        &           & 1,024 maps $15 \times 15$\\
   \hline
   conv6      & 1,024 maps $15 \times 15$        & 1,024 filters $3 \times 3$              & 1      & [1,1;1,1] &                          \\
   relu6      &                                  &                                         &        &           &                          \\
   pool6      &                                  & max on $[3,3]$ regions                  & 3      & 0         & 1,024 maps $5 \times 5$  \\
   \hline     
   fc7        & 1,024 maps $5 \times 5$          & 4,096 filters $5 \times 5$              & 1      &           & vector of size 4,096     \\
   relu7      &                                  &                                         &        &           & vector of size 4,096     \\
   \hline
   fc8        & vector of size 4,096             & 4,096 neurons                           &        &           & vector of size 4,096     \\ 
   relu8      & vector of size 4,096             &                                         &        &           & vector of size 4,096     \\
   \hline
   fc9        & vector of size 4,096             & 1,000 neurons                           &        &           & vector of size 1,000     \\
   softmax    & vector of size 1,000             &                                         &        &           & vector of size 1,000     \\
   ????       & vector of size 1,000             &                                         &        &           & loss value               \\
   \hline
 \end{tabular}
 \caption{Overfeat (large) architecture \cite{Sermanet14overfeat}. Total weights: .}
 \label{table:Overfeat}
\end{center}
\end{table} 





\subsection{CNN-M (Chatfield)}

In their paper \cite{Chatfield14}, Chatfield et al. propose a study comparing Fisher Vectors representations to deep representations on 
PASCALVOC2007. In this context, they present three architectures: CNN-F (fast but not the most accurate), CNN-M (medium, quite accurate), CNN-S 
(slow but not much more accurate than CNN-M). In this report, we describe the CNN-M architecture (which is used throughout \cite{Chatfield14} as 
the reference method). 

This network contains \textbf{102,897,440} learnable weights (6,526,752 in the convolutional part, 96,370,688 in the fully-connected part). It is 
worth noticing that the authors propose ``lighter'' versions of their network by reducing the size of fc7: CNN-M-2048 has 2048 neurons in fc7 (total 
weights: 92,460,832), CNN-M-1024 has 1024 neurons in fc7 (total weights: 87,242,528), CNN-M-128 has 128 neurons in fc7 (total weights: 82,676,512). 
CNN-M has however been widely used for comparative studies in other works \cite{}. 

This architecture was designed to classify ImageNet ILSVRC12 images \cite{Deng09ImageNet}: these 1,2 million images belong to 1,000 classes. 
The weights of this network have been learned on ILSVRC12 data and are online available\footnote{CNN-M pretrained weigths: 
\url{http://www.vlfeat.org/matconvnet/pretrained/}}.

It should be noted that CNN-M is quite similar to AlexNet (described in section \ref{AlexNet}). However, CNN-M is much larger (larger filter banks in 
conv3, conv4, conv5) and reduces much less available information in the first layers (in conv1, stride 2 for CNN-M vs. stride 4 in AlexNet).


\begin{table}[h]
\begin{center}
 \begin{tabular}{|l||l||l|c|c||l|}
   \hline
   Layer name & Input                            & Details                                 & Stride & Padding   & Output                   \\
   \hline
   \hline
   conv1      & $224 \times 224 \times 3$ images & 96 filters $7 \times 7 \times 3$        & 2      & 0         & 96 maps $109 \times 109$ \\
   relu1      & 96 maps $109 \times 109$         &                                         &        &           & 96 maps $109 \times 109$ \\
   norm1      & 96 maps $109 \times 109$         & contrast normalization                  &        &           & 96 maps $109 \times 109$ \\
   pool1      & 96 maps $109 \times 109$         & max on $[3,3]$ regions                  & 2      & 0         & 96 maps $54 \times 54$   \\
   \hline
   conv2      & 96 maps $54 \times 54$           & 256 filters $5 \times 5 \times 96$      & 2      & [1,1;1,1] & 256 maps $26 \times 26$  \\
   relu2      & 256 maps $26 \times 26$          &                                         &        &           & 256 maps $26 \times 26$  \\
   norm2      & 256 maps $26 \times 26$          & contrast normalization                  &        &           & 256 maps $26 \times 26$  \\
   pool2      & 256 maps $26 \times 26$          & max on $[3,3]$ regions                  & 2      & [0,1;0,1] & 256 maps $13 \times 13$  \\
   \hline
   conv3      & 256 maps $13 \times 13$          & 512 filters $3 \times 3 \times 256$     & 1      & [1,1;1,1] & 512 maps $13 \times 13$  \\
   relu3      & 512 maps $13 \times 13$          &                                         &        &           & 512 maps $13 \times 13$  \\
   \hline
   conv4      & 512 maps $13 \times 13$          & 512 filters $3 \times 3 \times 256$     & 1      & [1,1;1,1] & 512 maps $13 \times 13$  \\
   relu4      & 512 maps $13 \times 13$          &                                         &        &           & 512 maps $13 \times 13$  \\
   \hline
   conv5      & 512 maps $13 \times 13$          & 512 filters $3 \times 3 \times 256$     & 1      & [1,1;1,1] & 512 maps $13 \times 13$  \\
   relu5      & 512 maps $13 \times 13$          &                                         &        &           & 512 maps $13 \times 13$  \\
   pool5      & 512 maps $13 \times 13$          & max on $[3,3]$ regions                  & 2      & 0         & 512 maps $6 \times 6$    \\
   \hline
   fc6        & 512 maps $6 \times 6$            & 4,096 neurons                           &        &           & vector of size 4,096     \\
   relu6      & vector of size 4,096             &                                         &        &           & vector of size 4,096     \\
   \hline
   fc7        & vector of size 4,096             & 4,096 neurons                           &        &           & vector of size 4,096     \\
   relu7      & vector of size 4,096             &                                         &        &           & vector of size 4,096     \\
   \hline
   fc8        & vector of size 4,096             & 1,000 neurons                           &        &           & vector of size 1,000     \\
   softmax    & vector of size 1,000             &                                         &        &           & vector of size 1,000     \\
   log-loss   & vector of size 1,000             & only for training                       &        &           & loss value               \\
   \hline 
 \end{tabular}
 \caption{CNN-M architecture \cite{Chatfield14}. Total weights: 102,897,440.}
 \label{table:CIFAR}
\end{center}
\end{table} 









%===========================================================================================================================================

\section{Large networks}

\subsection{Very deep (Zisserman)}

\subsection{GoogLeNet}





